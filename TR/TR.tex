\documentclass[a4paper, 12pt]{report}
%%{{{--------PACK--------
\usepackage[T2A]{fontenc}
\usepackage{ucs}
\usepackage[utf8x]{inputenc}
\usepackage[english, russian]{babel}
\usepackage{hyperref}
\usepackage{amstext}
\usepackage{chngcntr}
%%}}}
\pagestyle{plain}

% ТЕХНИЧЕСКОЕ ЗАДАНИЕ НА СОЗДАНИЕ АВТОМАТИЗИРОВАННОЙ СИСТЕМЫ
% ГОСТ 34.602-89
% http://www.nist.ru/hr/doc/gost/34-602-89.htm

\begin{document}

\author{Дорофеев Андрей}
\title{Техническое задание на создание системы <<Интерпретатор коммандной оболочки для встраиваемых систем на базе 
микросхем семейства AVR>>}
\date{\today}
\maketitle

\tableofcontents

\counterwithout{section}{chapter}
\section{Общие сведения}
%%{{{----------------
Наименование системы -- командная оболочка (далее по тексту -- 
Система).

Плановый срок окончания работы по созданию системы -- март 2014 года. 
Время, отведенное на работу: 5 недель.

Результат работы по созданию системы будет оформлен в виде курсового
проекта по дисциплине <<проектирование трансляторов>>.
%%}}}

\section{Назначение и цели создания системы}
%%{{{----------------

\subsection{Назначение системы}

Система предназначена для установки на встраиваемые ЭВМ, такие как микроконтроллеры, с целью
упрощения отладки программного или аппаратного обеспечения, увеличения интерактивности устройства.

Система предназначена для решения перечисленных ниже задач:
\begin{enumerate}
	\item Интерпретация текстовых команд.
\end{enumerate}

\subsection{Цели создания системы}

Целями создания системы являются:
\begin{enumerate}
	\item увеличение скорости разработки;
	\item возможность изменять алгоритм работы устройства без перепрошивки;
	\item обеспечение интерактивности аппаратно-программного комплекса.
\end{enumerate}
%%}}}

\section{Характеристика объекта автоматизации}
%%{{{----------------
%%}}}

\section{Требования к системе}
%%{{{----------------

%---------------------------------------------------------------------------

\subsection{Требования к системе в целом}

\subsubsection{Требования к структуре и функционированию системы}

Система должна предоставлять возможность интерпретировать текстовые команды в ASCII-кодировке.
Среди конструкций языка должны присутствовать следующие:
\begin{enumerate}
	\item оператор присваивания {\tt set \_=\_};
	\item оператор разыменовывания {\tt \$\_} и {\tt \#\_};
	\item вызов заранее добавленной функции;
	\item операторы ветвления {\tt if, else, elif};
	\item оператор вычисления-подстановки {\tt ( \_ )};
		поддерживающий операции {\tt ==, !=, >, <, >=, <=, +, -, *, /, \%}.
\end{enumerate}

Пример -- несколько команд на специализированном языке:
\begin{itemize}
	\item {\tt echo this is string;}
	\item {\tt set\_port -p A --binary-value 00100100;}
	\item {\tt while ( \#time < 10 ) sleep 1; get\_time time; end}
	\item {\tt set var=middle; echo left \$var right;}
	\item {\tt if ( \$i == 1 ) sleep 1; elif (\$i == 2) sleep 2; else sleep 3; end}
\end{itemize}

Система <<командная оболочка>> должна иметь следующие подсистемы:
\begin{enumerate}
	\item модуль лексического анализа входящих команд
	\item модуль синтаксического анализа входящих команд
	\item модуль исполнения команд
\end{enumerate}

\subsubsection{Требования к численности и квалификации пользователя системы и режиму его работы}

\subsubsection{Показатели назначения}
{\bf Платформа AVR:}\\
Размер линкованного исполняемого кода не должен превышать 1Кб. 
Система должна быть рассчитана на 
исполнение на ЭВМ с 8-битной архитектурой, низкой тактовой частотой ЦПУ (около 1 МГц) и малым объемом 
памяти (около 2 Кб). \\
{\bf Платформа PC:}\\
Нет ограничений на объем используемой памяти и размер исполняемого файла. В этом случае предусматривается модуль-заглушка, имитирующий работу аппаратного обеспечения системы.
\subsubsection{Требования к надежности}

При возникновении ошибки требуется указать место (позицию в строке), причину; должны быть предприняты необходимые действия по восстановлению нормального режима функционирования. 
Пользователь должен получить исчерпывающую информацию о произошедшем.

\subsubsection{Требования безопасности}

\subsubsection{Требования к эргономике и технической эстетике}

\subsubsection{Требования к эксплуатации, техническому обслуживанию, ремонту и хранению компонентов системы}

\subsubsection{Требования к защите информации от несанкционированного доступа}

\subsubsection{Требования по сохранности информации при авариях}

\subsubsection{Требования к защите от влияния внешних воздействий}

\subsubsection{Требования к патентной чистоте}

Патентная чистота системы и ее частей должна быть обеспечена для Российской Федерации.

\subsubsection{Требования по стандартизации и унификации}

Допускается использование стандартной библиотеки языка C для платформы AVR 8-bit.
Исходный код системы должен быть переносимым в пределах этого семейства микроконтроллеров. Код должен быть пригоден 
для отладки логической составляющей на персональном компьютере без использования дополнительных аппаратных средств.

\subsubsection{Дополнительные требования}

Необходимо наличие возможности сборки интерпретатора под операционные системы для ПК, такие как Windows и GNU/Linux.

%---------------------------------------------------------------------------

\subsection{Требования к функциям (задачам), выполняемым системой}

Система должна обеспечивать выполнение перечисленных ниже функций:
\begin{enumerate}
	\item интерпретировать входные команды на специализированном языке
\end{enumerate}

%---------------------------------------------------------------------------

\subsection{Требования к видам обеспечения}

Для реализации программного кода следует использовать язык высокого уровня С. В качестве компилятора предполагается использовать AVR-GCC.

%%}}}

\section{Состав и содержание работ по созданию системы}
%%{{{----------------

Перечень стадий и этапов работ по созданию системы:\\
\begin{tabular}{|c|p{5in}|}
	\hline
	Дата & Этап \\
	\hline
	xx.xx.2014 & Исследование и обоснование создания АС \\
	\hline
	xx.xx.2014 & Написание технического задания \\
	\hline
	xx.xx.2014 & Разработка грамматики специализированного языка команд \\
	\hline
	xx.xx.2014 & Реализация программной части системы \\
	\hline
	xx.xx.2014 & Подготовка тестировочного стенда: необходимых модулей для системы и ПО для персонального компьютера \\
	\hline
	xx.xx.2014 & Проведение тестирования \\
	\hline

\end{tabular}
%%}}}

\section{Порядок контроля и приемки системы}
%%{{{----------------
\begin{enumerate}
	\item Демонстрация работы на тестовых примерах разработчика
	\item Приемочное тестирование на данных заказчика
	\item Ревизия исходного кода программы
	\item Проверка документации
\end{enumerate}
%%}}}

\section{Требования к составу и содержанию работ по подготовке объекта автоматизации к вводу системы в действие}
%%{{{----------------
В состав поставки должно входить:
\begin{itemize}
	\item техническое задание;
	\item пользовательский набор файлов:
		\begin{itemize}
			\item исполняемый файл;
			\item руководство;
			\item пример;
		\end{itemize}
	\item набор файлов для разработчика:
		\begin{itemize}
			\item техническое описание;
			\item исходные коды;
			\item необходимые библиотеки;
			\item примеры.
		\end{itemize}
\end{itemize}
%%}}}

\section{Требования к документированию}
%%{{{----------------

Необходимо разработать следующие виды документации:
\begin{enumerate}
\item Техническое задание
\item Техническое описание
\item Руководство пользователя
\end{enumerate}

\section{Источники разработки}

Системы-аналоги, на основе которых разрабатывалось ТЗ:
\begin{enumerate}
	\item GNU bash (https://www.gnu.org/software/bash/)
	\item avrsh (http://www.battledroids.net/downloads/avrsh.html)
\end{enumerate}
%%}}}
\end{document}

