\documentclass[a4paper, 12pt]{article}
%pack{{{
\usepackage[utf8]{inputenc}
\usepackage[russian]{babel}
\usepackage{amsmath}
\usepackage{pdfpages}
\usepackage{graphicx}
\usepackage{listings}
\usepackage{hyperref}
\usepackage{float}
\usepackage{longtable}
%}}}

\begin{document}

\author{Дорофеев Андрей}
\title{Методика тестирования}
\date{\today}
\maketitle

\tableofcontents

%{{{ ---------- SECTION: Введение ---------
\section{Введение}

%{{{ ---------- SUBSECTION: Наименование системы ---------
\subsection{Наименование системы} % (fold)

Полное наименование программного продукта: <<Интерпретатор коммандной оболочки для встраиваемых систем на базе 
микросхем семейства AVR>>. Продукт создается в рамках курсовой работы по дисциплине ``проектирование трансляторов''.

Производится тестирование версии на ПК. В данном документе не указаны 
дополнительные действия по тестированию компонентов системы на целевой аппаратной базе.

%}}}

%{{{ ---------- SUBSECTION: Назначение системы ---------
\subsection{Назначение системы} % (fold)

Система предназначена для записи на микросхемы и последующего использования
для ускорения разработки и отладки аппаратной части встраиваемой системы, создания 
единого протокола обмена данными между микроконтроллером и терминальным управляющим
устройством.

%}}}

%}}}

%{{{ ---------- SECTION: Ссылки на другие документы ---------
\section{Ссылки на другие документы}

%}}}

%{{{ ---------- SECTION: Подготовка к тестированию ---------
\section{Подготовка к тестированию}


%{{{ ---------- SUBSECTION: НАЗВАНИЕ ТЕСТА ---------
\subsection{Проверка запуска} % (fold)

С помощью данного теста устанавливается общая работоспособность системы.
\subsubsection{Подготовка аппаратной части} % (fold)
\subsubsection{Подготовка программной части} % (fold)
\begin{enumerate}
	\item Написание главного модуля для тестирования.
		Модуль должен осуществлять инициализацию компонентов 
		тестируемой системы. После этого необходимо совершить
		деинициализацию и выход со статусом 0;
	\item сборка исполняемого файла.
\end{enumerate}

%}}}

%{{{ ---------- SUBSECTION: НАЗВАНИЕ ТЕСТА ---------
\subsection{Проверка лексического анализатора} % (fold)

С помощью данного теста устанавливается работоспособность лексического анализатора: пропуск 
валидных токенов, отсеивание с сообщение об ошибке при обнаружении неверного токена.
\subsubsection{Подготовка аппаратной части} % (fold)
\subsubsection{Подготовка программной части} % (fold)
\begin{enumerate}
	\item Написание главного модуля, осуществляющего инициализацию
		тестируемой системы, анализ с помощью системы первого аргумента,
		представляемого при запуске и деинициализацию системы;
	\item сборка исполняемого файла;
	\item написание текстового файла, содержащего входные строки;
	\item написание скрипта, осуществляющего запуск собранного исполняемого файла
		с каждой строкой текстового файла с входными строками;
\end{enumerate}

%}}}

%{{{ ---------- SUBSECTION: НАЗВАНИЕ ТЕСТА ---------
\subsection{Проверка синтаксического анализатора} % (fold)

С помощью данного теста устанавливается работоспособность синтаксического анализатора:
способность распознавать принадлежность строки специализированному языку.
\subsubsection{Подготовка аппаратной части} % (fold)
\subsubsection{Подготовка программной части} % (fold)
\begin{enumerate}
	\item Написание главного модуля, осуществляющего инициализацию
		тестируемой системы, анализ с помощью системы первого аргумента,
		представляемого при запуске и деинициализацию системы;
	\item сборка исполняемого файла;
	\item написание текстового файла, содержащего входные строки;
	\item написание скрипта, осуществляющего запуск собранного исполняемого файла
		с каждой строкой текстового файла с входными строками;
\end{enumerate}

%}}}

%{{{ ---------- SUBSECTION: НАЗВАНИЕ ТЕСТА ---------
\subsection{Проверка модуля исполнения команд} % (fold)

С помощью данного теста устанавливается работоспособность модуля распознавания команд:
корректность ведения таблиц переменных и функций, правильность работы с переменными и пользовательскими функциями.
\subsubsection{Подготовка аппаратной части} % (fold)
\subsubsection{Подготовка программной части} % (fold)
\begin{enumerate}
	\item Написание главного модуля, осуществляющего инициализацию
		тестируемой системы, проведение доступных операций над 
		таблицами переменных и функций напрямую
		и деинициализацию системы;
	\item Написание главного модуля 1, осуществляющего инициализацию
		тестируемой системы, анализ с помощью системы первого аргумента,
		представляемого при запуске и деинициализацию системы;
	\item Написание главного модуля 2, осуществляющего инициализацию
		тестируемой системы, добавление собственной функции в таблицу, 
		анализ с помощью системы первого аргумента,
		представляемого при запуске и деинициализацию системы;
	\item сборка исполняемых файлов 1 и 2;
	\item написание текстового файла, содержащего входные строки;
	\item написание скриптов, осуществляющего запуск собранных исполняемых файлов
		с каждой строкой текстовых файлов с входными строками:
		\begin{itemize}
			\item для исп. файла 1, вызов несущ. функции
			\item для исп. файла 1, вызов сущ. функции
			\item для исп. файла 2, вызов добавленной функции
		\end{itemize}
\end{enumerate}

%}}}

%}}}

%{{{ ---------- SECTION: Описание тестов ---------
\section{Описание тестов}

%%{{{--------TEST _ N--------
\subsection{Проверка запуска} % (fold)
\subsubsection{Инициализация и деинициализация системы} % (fold)

\begin{tabular}{|l|p{4in}|}
	\hline
	Тестируемые требования	&Общая работоспособность.\\
	\hline
	Необходимые предпосылки	&\\
	\hline
	Входы			&\\
	\hline
	Ожидаемые результаты	&Программа заканчивает выполнение без сообщений об ошибках со статусом 0.\\
	\hline
	Критерии соответствия	&Статус при выходе 0.\\
	\hline
	Процедура тестирования	&Запуск исполняемого файла. Проверка статуса завершения.\\
	\hline
	Ограничения		&\\
	\hline
\end{tabular}

%%}}}
%%{{{--------TEST _ N--------
\subsection{Проверка лексического анализатора} % (fold)

\subsubsection{Корректная входная строка} % (fold)

\begin{tabular}{|l|p{4in}|}
	\hline
	Тестируемые требования	&``Система должна предоставлять возможность интерпретировать текстовые
	команды в ASCII-кодировке''\\
	\hline
	Необходимые предпосылки	&\\
	\hline
	Входы			&текстовый файл, содержащий корректные с лексической точки
	зрения строки, содержащие все известные токены\\
	\hline
	Ожидаемые результаты	&Программа завершает работу без сообщений об ошибках и со статусом 0\\
	\hline
	Критерии соответствия	&Статус выхода 0\\
	\hline
	Процедура тестирования	&Запуск скрипта\\
	\hline
	Ограничения		&\\
	\hline
\end{tabular}

\subsubsection{Некорректная входная строка} % (fold)

\begin{tabular}{|l|p{4in}|}
	\hline
	Тестируемые требования	&``Система должна предоставлять возможность интерпретировать текстовые
	команды в ASCII-кодировке''\\
	\hline
	Необходимые предпосылки	&\\
	\hline
	Входы			&текстовый файл, содержащий некорректные с лексической точки
	зрения строки\\
	\hline
	Ожидаемые результаты	&Программа завершает работу с сообщением об ошибке и 
	указанием места во входной строке\\
	\hline
	Критерии соответствия	&Статус выхода 1, \newline сообщение об ошибке\\
	\hline
	Процедура тестирования	&Запуск скрипта\\
	\hline
	Ограничения		&\\
	\hline
\end{tabular}

%%}}}
%%{{{--------TEST _ N--------
\subsection{Проверка синтаксического анализатора} % (fold)

\subsubsection{Корректная входная строка} % (fold)

\begin{tabular}{|l|p{4in}|}
	\hline
	Тестируемые требования	&``Система должна предоставлять возможность интерпретировать текстовые
	команды в ASCII-кодировке''\\
	\hline
	Необходимые предпосылки	&\\
	\hline
	Входы			&текстовый файл, содержащий строки на специализированном языке.\newline
				Строки в совокупности должны покрывать все синтаксические конструкции языка.\\
	\hline
	Ожидаемые результаты	&Сообщений об ошибках не выводится. Статусы выхода 0\\
	\hline
	Критерии соответствия	&Статусы выхода 0\\
	\hline
	Процедура тестирования	&Запуск скрипта\\
	\hline
	Ограничения		&\\
	\hline
\end{tabular}


\subsubsection{Некорректная входная строка 1} % (fold)
Пропуск разделителя ({\tt ;}).

\begin{table}[H]
	\centering

	\begin{tabular}{|l|p{4in}|}
		\hline
		Тестируемые требования	&``Система должна предоставлять возможность интерпретировать текстовые
		команды в ASCII-кодировке''\\
		\hline
		Необходимые предпосылки	&\\
		\hline
		Входы			&текстовый файл, содержащий строки на специализированном языке.\newline
		Строки содержат единственный тип ошибок --- пропущена точка с запятой.\\
		\hline
		Ожидаемые результаты	&Сообщения об ошибках. Статусы выхода -1\\
		\hline
		Критерии соответствия	&Статусы выхода -1\newline
		Сообщения о неожиданном токене с указанием корректной позиции 
		пропущенной точки с запятой.\\
		\hline
		Процедура тестирования	&Запуск скрипта\\
		\hline
		Ограничения		&\\
		\hline
	\end{tabular}

\end{table}

\subsubsection{Некорректная входная строка 2} % (fold)
Неожиданные символы в {\tt control}: {\tt +,-,=,>,<,\%,\&,!,\#}.
	\begin{table}[H]
		\centering
		\begin{tabular}{|l|p{4in}|}
			\hline
			Тестируемые требования	&``Система должна предоставлять возможность интерпретировать текстовые
			команды в ASCII-кодировке''\\
			\hline
			Необходимые предпосылки	&\\
			\hline
			Входы			&текстовый файл, содержащий строки на специализированном языке.\newline
			Строки содержат единственный тип ошибок --- перечисленные выше символы находятся
			в конструкции языка {\tt control}.\\
			\hline
			Ожидаемые результаты	&Сообщения об ошибках. Статусы выхода -1\\
			\hline
			Критерии соответствия	&Статусы выхода -1\newline
			Сообщения о неожиданном токене с указанием корректной позиции 
			с неверным токеном.\\
			\hline
			Процедура тестирования	&Запуск скрипта\\
			\hline
			Ограничения		&\\
			\hline
		\end{tabular}
	\end{table}

	\subsubsection{Некорректная входная строка 3} % (fold)
	Отсутствующий токен {\tt end} в конце {\tt if, while}.

	\begin{table}[H]
		\centering
		\begin{tabular}{|l|p{4in}|}
			\hline
			Тестируемые требования	&``Система должна предоставлять возможность интерпретировать текстовые
			команды в ASCII-кодировке''\\
			\hline
			Необходимые предпосылки	&\\
			\hline
			Входы			&текстовый файл, содержащий строки на специализированном языке.\newline
			Строки содержат единственный тип ошибок --- 
			отсутствующий токен {\tt end} в конце {\tt if, while}\\
			\hline
			Ожидаемые результаты	&Сообщения об ошибках. Статусы выхода -1\\
			\hline
			Критерии соответствия	&Статусы выхода -1\newline
			Сообщения о неожиданном токене с указанием корректной позиции 
			с неверным токеном.\\
			\hline
			Процедура тестирования	&Запуск скрипта\\
			\hline
			Ограничения		&\\
			\hline
		\end{tabular}
	\end{table}

\subsubsection{Некорректная входная строка 3} % (fold)
Отсутствующий токен {\tt end} в конце {\tt if, while}.

	\begin{table}[H]
		\centering
		\begin{tabular}{|l|p{4in}|}
			\hline
			Тестируемые требования	&``Система должна предоставлять возможность интерпретировать текстовые
			команды в ASCII-кодировке''\\
			\hline
			Необходимые предпосылки	&\\
			\hline
			Входы			&текстовый файл, содержащий строки на специализированном языке.\newline
			Строки содержат единственный тип ошибок --- 
			отсутствующий токен {\tt end} в конце {\tt if, while}\\
			\hline
			Ожидаемые результаты	&Сообщения об ошибках. Статусы выхода -1\\
			\hline
			Критерии соответствия	&Статусы выхода -1\newline
			Сообщения о неожиданном токене с указанием корректной позиции с неверным токеном.\\
			\hline
			Процедура тестирования	&Запуск скрипта\\
			\hline
			Ограничения		&\\
			\hline
		\end{tabular}
	\end{table}

	\subsubsection{Некорректная входная строка 4} % (fold)
	Ошибки в рамках {\tt eval}: {\tt =} вместо {\tt ==}, незакрытая скобка,
	пропущенный операнд, пропущенный оператор.

	\begin{table}[H]
		\centering
		\begin{tabular}{|l|p{4in}|}
			\hline
			Тестируемые требования	&``Система должна предоставлять возможность интерпретировать текстовые
			команды в ASCII-кодировке''\\
			\hline
			Необходимые предпосылки	&\\
			\hline
			Входы			&текстовый файл, содержащий строки на специализированном языке.\newline
			Строки содержат перечисленные выше ошибки\\
			\hline
			Ожидаемые результаты	&Сообщения об ошибках. Статусы выхода -1\\
			\hline
			Критерии соответствия	&Статусы выхода -1\newline
			Сообщения о неожиданном токене с указанием корректной позиции 
			с неверным токеном.\\
			\hline
			Процедура тестирования	&Запуск скрипта\\
			\hline
			Ограничения		&\\
			\hline
		\end{tabular}
	\end{table}
%%}}}
%%{{{--------TEST _ N--------
\subsection{Проверка модуля исполнения команд} % (fold)
\subsubsection{Корректность ведения таблиц переменных и функций} % (fold)

\begin{table}[H]
	\centering
	\begin{tabular}{|l|p{4in}|}
		\hline
		Тестируемые требования	&``Система должна предоставлять возможность интерпретировать текстовые
		команды в ASCII-кодировке''\\
		\hline
		Необходимые предпосылки	&\\
		\hline
		Входы			&\\
		\hline
		Ожидаемые результаты	&Статус завершения 0\\
		\hline
		Критерии соответствия	&Статусы выхода 0\newline\\
		\hline
		Процедура тестирования	&Запуск исполняемого файла\\
		\hline
		Ограничения		&\\
		\hline
	\end{tabular}
\end{table}

\subsubsection{Корректность работы с переменными и функциями 1} % (fold)
Оператор присваивания.

\begin{table}[H]
	\centering
	\begin{tabular}{|l|p{4in}|}
		\hline
		Тестируемые требования	&``Система должна предоставлять возможность интерпретировать текстовые
		команды в ASCII-кодировке''\\
		\hline
		Необходимые предпосылки	&\\
		\hline
		Входы			&текстовый файл с входными стоками,\newline
		строки содержат синтаксически корректные командные последовательности,
		разное количество присваиваний: 1-3. Повторное присваивание. Каждый шаг
		должен сопровождаться командой {\tt echo} для проверки результата.\\
		\hline
		Ожидаемые результаты	&Статус завершения 0, вывод команды соответствует присвоениям\\
		\hline
		Критерии соответствия	&Статусы выхода 0\newline корректные значения переменных\\
		\hline
		Процедура тестирования	&Запуск исполняемого файла, \newline проверка выведенных значений\\
		\hline
		Ограничения		&\\
		\hline
	\end{tabular}
\end{table}

\subsubsection{Корректность работы с переменными и функциями 2} % (fold)
Операторы разыменовывания.

\begin{table}[H]
	\centering
	\begin{tabular}{|l|p{4in}|}
		\hline
		Тестируемые требования	&``Система должна предоставлять возможность интерпретировать текстовые
		команды в ASCII-кодировке''\\
		\hline
		Необходимые предпосылки	&\\
		\hline
		Входы			&текстовый файл с входными стоками,\newline
		строки содержат синтаксически корректные командные последовательности:
		\begin{enumerate}
			\item разыменовывание необъявленной переменной
			\item разыменовывание объявленной до этого переменной
			\item два разыменовывания подряд
			\item разыменовывание в строку в {\tt eval}
			\item разыменовывание в число в {\tt eval}
		\end{enumerate}
		Разыменовывание происходит агрументом функции {\tt echo}.\\
		\hline
		Ожидаемые результаты	&Статус завершения 0\newline
		\begin{enumerate}
			\item ошибка о необъявленной переменной с указанием на токен
			\item выводится корректное значение
			\item выводится корректное значение (два раза)
			\item выводится корректное значение
			\item выводится корректное значение
		\end{enumerate}
		\\
		\hline
		Критерии соответствия	&Статусы выхода 0\newline корректные значения переменных\\
		\hline
		Процедура тестирования	&Запуск исполняемого файла, \newline проверка выведенных значений\\
		\hline
		Ограничения		&\\
		\hline
	\end{tabular}
\end{table}

\subsubsection{Корректность работы с переменными и функциями 3} % (fold)
Вызовы функций.

\begin{table}[H]
	\centering
	\begin{tabular}{|l|p{4in}|}
		\hline
		Тестируемые требования	&``Система должна предоставлять возможность интерпретировать текстовые
		команды в ASCII-кодировке''\\
		\hline
		Необходимые предпосылки	&\\
		\hline
		Входы			&текстовый файл с входными стоками,\newline
		строки содержат синтаксически корректные командные последовательности:
		\begin{enumerate}
			\item вызов функции, которой нет в таблице
			\item вызов функции, которая есть в таблице по умолчанию ({\tt echo})
			\item вызов собственной функции, которая заранее добавлена в таблицу
		\end{enumerate}\\
		\hline
		Ожидаемые результаты	&Статус завершения 0\newline
		\begin{enumerate}
			\item сообщение о неизвестной функции с указанием на соответствующий токен
			\item корректный вывод аргументов
			\item корректное выполнение новой функции
		\end{enumerate}
		\\
		\hline
		Критерии соответствия	& \begin{enumerate}
			\item сообщение об ошибке, статус -1
			\item нет сообщений об ошибках, статус 0, выведенные значения соотв. агрументам
			\item нет сообщений об ошибках, статус 0, произведены действия, предусмотренные функцией
		\end{enumerate}
		\\
		\hline
		Процедура тестирования	&
		\begin{enumerate}
			\item запуск скрипта 1 для исполняемого файла 1, проверка вывода, проверка статуса
			\item запуск скрипта 2 для исполняемого файла 1, проверка вывода, проверка статуса
			\item запуск скрипта 1 для исполняемого файла 2, проверка вывода, проверка статуса
		\end{enumerate}\\
		\hline
		Ограничения		&\\
		\hline
	\end{tabular}
\end{table}

\subsubsection{Корректность работы операторов в блоке {\tt eval}} % (fold)
Проверка типов. Корректность выполнения.

\begin{longtable}[H]{|l|p{4in}|}
	%\centering
		\hline
		Тестируемые требования	&``Система должна предоставлять возможность интерпретировать текстовые
		команды в ASCII-кодировке''\\
		\hline
		Необходимые предпосылки	&\\
		\hline
		Входы			&текстовый файл с входными стоками,\newline
		строки содержат синтаксически корректные командные последовательности:
		\begin{enumerate}
			\item {\tt echo ( 5 == 'five');}
			\item {\tt echo ( 5 >= 'five');}
			\item {\tt echo ( 5 <= 'five');}
			\item {\tt echo ( 5 > 'five');}
			\item {\tt echo ( 5 < 'five');}
			\item {\tt echo ( 5 - 'five');}
			\item {\tt echo ( 5 + 'five');}
			\item {\tt echo ( 5 * 'five');}
			\item {\tt echo ( 5 \% 'five');}
			\item {\tt echo (5 == 5);}
			\item {\tt echo (5 == 6);}
			\item {\tt echo ('str' == 'str');}
			\item {\tt echo ('str' == 'str1');}
			\item {\tt echo (5 != 5);}
			\item {\tt echo (5 != 6);}
			\item {\tt echo ('str' != 'str');}
			\item {\tt echo ('str' != 'str1');}
			\item {\tt echo (6 + 6);}
			\item {\tt echo (6 - 6);}
		\end{enumerate}\\&
				\begin{enumerate}
						\setcounter{enumi}{19}
			\item {\tt echo (6 * 6);}
			\item {\tt echo (5 \% 6);}
			\item {\tt echo (6 / 2);}
			\item {\tt echo (6 > 2);}
			\item {\tt echo (6 > 6);}
			\item {\tt echo (6 < 6);}
			\item {\tt echo (6 < 2);}
			\item {\tt echo (6 >= 2);}
			\item {\tt echo (6 >= 6);}
			\item {\tt echo (6 <= 2);}
			\item {\tt echo (6 <= 6);}
		\end{enumerate}\\
		\hline
		Ожидаемые результаты	&Статус завершения 0\newline
		\begin{enumerate}
			\item {\tt echo ( 5 == 'five');} $\rightarrow$ ошибка типов
			\item {\tt echo ( 5 >= 'five');} $\rightarrow$ ошибка типов
			\item {\tt echo ( 5 <= 'five');} $\rightarrow$ ошибка типов
			\item {\tt echo ( 5 > 'five');} $\rightarrow$ ошибка типов
			\item {\tt echo ( 5 < 'five');} $\rightarrow$ ошибка типов
			\item {\tt echo ( 5 - 'five');} $\rightarrow$ ошибка типов
			\item {\tt echo ( 5 + 'five');} $\rightarrow$ ошибка типов
			\item {\tt echo ( 5 * 'five');} $\rightarrow$ ошибка типов
			\item {\tt echo ( 5 \% 'five');} $\rightarrow$ ошибка типов
			\item {\tt echo (5 == 5);} $\rightarrow$ <<t>>
			\item {\tt echo (5 == 6);} $\rightarrow$ <<t>>
			\item {\tt echo ('str' == 'str');} $\rightarrow$ <<t>>
			\item {\tt echo ('str' == 'str1');} $\rightarrow$ <<t>>
			\item {\tt echo (5 != 5);} $\rightarrow$ <<f>>
			\item {\tt echo (5 != 6);} $\rightarrow$ <<t>>
			\item {\tt echo ('str' != 'str');} $\rightarrow$ <<f>>
			\item {\tt echo ('str' != 'str1');} $\rightarrow$ <<t>>
			\item {\tt echo (6 + 6);} $\rightarrow$ <<12>>
			\item {\tt echo (6 - 6);} $\rightarrow$ <<0>>
		\end{enumerate}\\&
				\begin{enumerate}
						\setcounter{enumi}{19}
			\item {\tt echo (6 * 6);} $\rightarrow$ <<36>>
			\item {\tt echo (5 \% 6);} $\rightarrow$ <<5>>
			\item {\tt echo (6 / 2);} $\rightarrow$ <<3>>
			\item {\tt echo (6 > 2);} $\rightarrow$ <<t>>
			\item {\tt echo (6 > 6);} $\rightarrow$ <<f>>
			\item {\tt echo (6 < 6);} $\rightarrow$ <<f>>
			\item {\tt echo (6 < 2);} $\rightarrow$ <<f>>
			\item {\tt echo (6 >= 2);} $\rightarrow$ <<t>>
			\item {\tt echo (6 >= 6);} $\rightarrow$ <<t>>
			\item {\tt echo (6 <= 2);} $\rightarrow$ <<f>>
			\item {\tt echo (6 <= 6);} $\rightarrow$ <<t>>
		\end{enumerate}
		\\
		\hline
		Критерии соответствия	&
		\begin{itemize}
			\item отсутствие ошибок, там, где типы соответствуют
			\item наличие соотв. ошибок, там, где типы не соответствуют
			\item соответствие вывода ожидаемому
		\end{itemize}
		\\
		\hline
		Процедура тестирования	&
		\begin{enumerate}
			\item запуск скрипта 1 для исполняемого файла 1, проверка вывода, проверка статуса
			\item запуск скрипта 2 для исполняемого файла 1, проверка вывода, проверка статуса
			\item запуск скрипта 1 для исполняемого файла 2, проверка вывода, проверка статуса
		\end{enumerate}\\
		\hline
		Ограничения		&\\
		\hline
	\end{longtable}
%%}}}

%}}}

\end{document}
