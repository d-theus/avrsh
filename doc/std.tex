\documentclass[a4paper, 12pt]{article}
%pack{{{
\usepackage[utf8]{inputenc}
\usepackage[russian]{babel}
\usepackage{amsmath}
\usepackage{pdfpages}
\usepackage{graphicx}
\usepackage{listings}
\usepackage{hyperref}
%}}}

\begin{document}

\author{Дорофеев Андрей}
\title{Методика тестирования}
\date{\today}
\maketitle

\tableofcontents

%{{{ ---------- SECTION: Введение ---------
\section{Введение}

%{{{ ---------- SUBSECTION: Наименование системы ---------
\subsection{Наименование системы} % (fold)

Полное наименование программного продукта: <<Интерпретатор коммандной оболочки для встраиваемых систем на базе 
микросхем семейства AVR>>. Продукт создается в рамках курсовой работы по дисциплине ``проектирование трансляторов''.

Производится тестирование версии на ПК. В данном документе не указаны 
дополнительные действия по тестированию компонентов системы на целевой аппаратной базе.

%}}}

%{{{ ---------- SUBSECTION: Назначение системы ---------
\subsection{Назначение системы} % (fold)

Система предназначена для записи на микросхемы и последующего использования
для ускорения разработки и отладки аппаратной части встраиваемой системы, создания 
единого протокола обмена данными между микроконтроллером и терминальным управляющим
устройством.

%}}}

%}}}

%{{{ ---------- SECTION: Ссылки на другие документы ---------
\section{Ссылки на другие документы}

%}}}

%{{{ ---------- SECTION: Подготовка к тестированию ---------
\section{Подготовка к тестированию}


%{{{ ---------- SUBSECTION: НАЗВАНИЕ ТЕСТА ---------
\subsection{Проверка запуска} % (fold)

С помощью данного теста устанавливается общая работоспособность системы.
\subsubsection{Подготовка аппаратной части} % (fold)
\subsubsection{Подготовка программной части} % (fold)
\begin{enumerate}
	\item Написание главного модуля для тестирования.
		Модуль должен осуществлять инициализацию компонентов 
		тестируемой системы. После этого необходимо совершить
		деинициализацию и выход со статусом 0;
	\item сборка исполняемого файла.
\end{enumerate}

%}}}

%{{{ ---------- SUBSECTION: НАЗВАНИЕ ТЕСТА ---------
\subsection{Проверка лексического анализатора} % (fold)

С помощью данного теста устанавливается работоспособность лексического анализатора: пропуск 
валидных токенов, отсеивание с сообщение об ошибке при обнаружении неверного токена.
\subsubsection{Подготовка аппаратной части} % (fold)
\subsubsection{Подготовка программной части} % (fold)
\begin{enumerate}
	\item Написание главного модуля, осуществляющего инициализацию
		тестируемой системы, анализ с помощью системы первого аргумента,
		представляемого при запуске и деинициализацию системы;
	\item сборка исполняемого файла;
	\item написание текстового файла, содержащего входные строки;
	\item написание скрипта, осуществляющего запуск собранного исполняемого файла
		с каждой строкой текстового файла с входными строками;
\end{enumerate}

%}}}

%{{{ ---------- SUBSECTION: НАЗВАНИЕ ТЕСТА ---------
\subsection{Проверка синтаксического анализатора} % (fold)

С помощью данного теста устанавливается работоспособность синтаксического анализатора:
способность распознавать принадлежность строки специализированному языку.
\subsubsection{Подготовка аппаратной части} % (fold)
\subsubsection{Подготовка программной части} % (fold)
\begin{enumerate}
	\item Написание главного модуля, осуществляющего инициализацию
		тестируемой системы, анализ с помощью системы первого аргумента,
		представляемого при запуске и деинициализацию системы;
	\item сборка исполняемого файла;
	\item написание текстового файла, содержащего входные строки;
	\item написание скрипта, осуществляющего запуск собранного исполняемого файла
		с каждой строкой текстового файла с входными строками;
\end{enumerate}

%}}}

%{{{ ---------- SUBSECTION: НАЗВАНИЕ ТЕСТА ---------
\subsection{Проверка модуля исполнения команд} % (fold)

С помощью данного теста устанавливается работоспособность модуля распознавания команд:
корректность ведения таблиц переменных и функций, правильность работы с переменными и пользовательскими функциями.
способность распознавать принадлежность строки специализированному языку.
\subsubsection{Подготовка аппаратной части} % (fold)
\subsubsection{Подготовка программной части} % (fold)
\begin{enumerate}
	\item Написание главного модуля, осуществляющего инициализацию
		тестируемой системы, проведение доступных операций над 
		таблицами переменных и функций напрямую
		и деинициализацию системы;
	\item Написание главного модуля, осуществляющего инициализацию
		тестируемой системы, анализ с помощью системы первого аргумента,
		представляемого при запуске и деинициализацию системы;
	\item сборка исполняемых файлов;
	\item написание текстового файла, содержащего входные строки;
	\item написание скрипта, осуществляющего запуск собранного исполняемого файла
		с каждой строкой текстового файла с входными строками;
\end{enumerate}

%}}}

%}}}

%{{{ ---------- SECTION: Описание тестов ---------
\section{Описание тестов}

%%{{{--------TEST _ N--------
\subsection{Проверка запуска} % (fold)
\subsubsection{Инициализация и деинициализация системы} % (fold)

\begin{tabular}{|c|c|}
	Тестируемые требования	 &Общая работоспособность.\\
	Необходимые предпосылки	 &Программа заканчивает выполнение без сообщений об ошибках со статусом 0.\\
	Входы			 &
	Ожидаемые результаты	 &Запуск исполняемого файла. Проверка статуса завершения.\\
	Критерии соответствия	 &Статус при выходе 0.\\
	Процедура тестирования	 &\\
	Ограничения
\end{tabular}

\begin{description}
	\item[] 
	\item[]
	\item[] %названия, задачи, описания
	\item[]
		
	\item[]
		
	\item[]
		
	\item[Ограничения]
\end{description}

%%}}}
%%{{{--------TEST _ N--------
\subsection{Проверка лексического анализатора} % (fold)
\subsubsection{Название кейса} % (fold)


\begin{description}
	\item[Тестируемые требования] 
	\item[Необходимые предпосылки] 
	\item[Входы] %названия, задачи, описания
	\item[Тестируемые требования] 
	\item[Ожидаемые результаты]
	\item[Критерии соответствия]
	\item[Процедура тестирования] item
	\item[Ограничения] item
\end{description}

%%}}}
%%{{{--------TEST _ N--------
\subsection{Проверка синтаксического анализатора} % (fold)
\subsubsection{Название кейса} % (fold)


\begin{description}
	\item[Тестируемые требования] 
	\item[Необходимые предпосылки] 
	\item[Входы] %названия, задачи, описания
	\item[Тестируемые требования] 
	\item[Ожидаемые результаты]
	\item[Критерии соответствия]
	\item[Процедура тестирования] item
	\item[Ограничения] item
\end{description}

%%}}}
%%{{{--------TEST _ N--------
\subsection{Проверка модуля исполнения команд} % (fold)
\subsubsection{Название кейса} % (fold)


\begin{description}
	\item[Тестируемые требования] 
	\item[Необходимые предпосылки] 
	\item[Входы] %названия, задачи, описания
	\item[Тестируемые требования] 
	\item[Ожидаемые результаты]
	\item[Критерии соответствия]
	\item[Процедура тестирования] item
	\item[Ограничения] item
\end{description}

%%}}}

%}}}

\end{document}
